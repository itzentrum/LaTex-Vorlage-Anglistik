% KOMA-Script Klasse für Report Din-A4 Papiergröße, 12pt Schriftgröße
\documentclass[12pt]{scrreprt}
% Paket für deutsche und englische Formate, Übersetzungen und Worttrennung 
% (Standardmäßig deutsch)
\usepackage [english,ngerman]{babel}
% Für geschachtelte Zitate
\usepackage[autostyle]{csquotes}
% Paket für Bedingungen (siehe unten) in *LaTeX
\usepackage{iftex}
% Pakete u.a. für die Darstellung von Umlauten/Sonderzeichen
\ifPDFTeX
   \usepackage[utf8]{inputenc}
   \usepackage[T1]{fontenc}
   \usepackage{lmodern}
\else
   \ifXeTeX
     \usepackage{xltxtra}
   \else 
     \usepackage{luatextra}
   \fi
   \defaultfontfeatures{Ligatures=TeX}
\fi
% Eineinhalbfacher Zeilenabstand
\usepackage[onehalfspacing]{setspace}
% Papiergröße und Ränder
\usepackage[a4paper, left=4cm, right=3cm, top=2.5cm, bottom=2.5cm]{geometry}
% Zugriff auf Titelinformationen
\usepackage{titling}
% Erweiterte Quotation-Umgebungen (small ist ~11pt bei 12pt im Dokument)
\usepackage[leftmargin=2.5cm,font={small,singlespacing}]{quoting}
% Für das Setzen von Versen
\usepackage{verse}
% Für die Bibliographie
% Der Chicago-Stil benötigt eine sehr aktuelle BibLaTex-Version (aktuell 2.8a)
% Üblicherweise muss für ein funktionierendes Literaturverzeichnis zunächst
% *LaTeX ausgeführt werden (muss bei jeder Änderungen der Verweise neu ausgeführt werden
% und erstellt eine .bcf-Datei), dann Biber mit der so entstandenen .bcf-Datei (muss bei jeder Änderung
% in der Bibliographie neu ausgeführt werden) und zum Schluss nocheinmal *LaTeX
\usepackage[backend=biber,style=chicago-authordate,language=english,maxnames=5,cmsdate=off]{biblatex}
\addbibresource{Vorlage_Anglistik.bib}

% Zufälliger Blindtext (kann für die eigentliche Arbeit entfernt werden)
\usepackage{blindtext}

% Linker Abstand bei Aufzählungen
\setlength{\leftmargini}{ 2.5cm }

% Folgende Felder für die Arbeit anpassen
\title{ (Titel der \\[0.5em] Seminararbeit) }
\author{ (Autor) }
\def \semester { (Semester) }
\def \veranstaltungstyp { (Veranstaltungstyp }
\def \veranstaltungstitel { Veranstaltungstitel) }
\def \dozent { (Dozent/in) }
\def \adresse { (Adresse, }
\def \telefon { Telefonnummer }
\def \mail { E-Mail) }
\def \studienfaecher { (Fachverbindung mit Angaben HF/NF) }
\def \fachsemester { (Fachsemester) }

% Eingebundene Dateien anzeigen
%\listfiles
% Beginn des eigentlichen Dokuments
\begin{document}

% Titelseite muss nicht direkt geändert werden, sondern nur die
% obigen Felder
\begin{titlingpage}
% Absatzbeginn nicht einrücken
\noindent
{\sffamily{\large
Ludwig-Maximilians-Universität München \\
Department für Anglistik und Amerikanistik \\
\semester \\
\veranstaltungstyp, \veranstaltungstitel \\
\dozent \\[0.2\textheight]
}
\begin{center}
{\Huge \thetitle } \\[0.1\textheight]
{\Large \theauthor } \\[0.05\textheight]
{\large
\adresse \\ 
Tel.: \telefon \\ 
E-Mail: \mail \\
\studienfaecher \\
Fachsemester: \fachsemester
}
\end{center}
}
\end{titlingpage}

% Inhaltsverzeichnis erstellen (*LaTex muss üblicherweise
% zwei mal ausgeführt werden, bis das Verzeichnis aktualisiert
% wird)
\tableofcontents

% Seitenzahl für das Inhaltsverzeichnis auf 1 setzen
\clearpage \setcounter{page}{1} 
\chapter{Kapitel}
\section{Abschnitt}
\subsection{Deutscher Unterabschnitt}
\blindtext
\subsection{Englischer Unterabschnitt}
% Sprache auf Englisch ändern
\begin{otherlanguage}{english}
% Mehrzeiliges, eingerücktes Zitat
\begin{quoting}
\blindtext
\end{quoting}
\end{otherlanguage}
\subsection{Wieder deutscher Unterabschnitt}
\blindtext
% Beispiel für ein mehrweiliges, abgesetzt zitiertes Gedicht/Lied
\subsection{Verse-Umgebung}
\settowidth{\versewidth}{In a cavern, in a canyon,}
\poemtitle*{Clementine}
\begin{verse}[\versewidth]
\poemlines{2}
\begin{altverse}
\flagverse{1.} In a cavern, in a canyon, \\
Excavating for a mine, \\
Lived a miner, forty-niner, \label{vs:49} \\
And his daughter, Clementine. \\!
\end{altverse}

\begin{altverse}
\flagverse{\textsc{chorus}} Oh my darling, Oh my darling, \\
Oh my darling Clementine. \\
Thou art lost and gone forever, \\
Oh my darling Clementine
\end{altverse}

\end{verse}
% Abschnitt verschiedenster Zitierweisen
\subsection{Zitierweisen}
Liste von Zitaten aus Sekundärquellen: Nummer 1 \parencite[][56-59]{Culler1997}, Nummer 2 \parencite[][103]{Baugh2002} , Nummer 3 \cite*[][]{Schabert2000}, Nummer 4 \Cite[][]{Crenshaw1995} , Nummer 5\footcite[][2: 117]{Huehn1995}, Nummer 6 \textcite[][290-297]{Kastovsky1992}, Nummer 7 \parencite[][587]{Wimsatt1959},  Nummer 8 \parencite[][5]{Jones1989} und Nummer 8 \parencite{Willey2003}

Liste von Zitaten aus Primärquellen: Nummer 1\footcite[][35]{Banville2005}, Nummer 2\footcite[][56]{Austen1813}. Nummer 3\footcite[][72]{Wrenn1996}, Nummer 4\footcite[][117]{McEwan1975}, Nummer 5\footcite[][1. 11-16]{Eliot2000}, Nummer 6\footcite[][1.5.189-190]{Shakespeare1985} und Nummer 7\footcite[][1:14:12]{Coppola1992}
% Die Bibliographie aufgeteilt anhand von Schlüsselwörtern, die auch im keyword-Feld der BibTex-Datei verwendet werden müssen
\chapter{Bibliographie}
\printbibliography[keyword=Primaerquelle,heading=subbibliography,title={Primärquellen}]
\printbibliography[keyword=Sekundaerquelle,heading=subbibliography,title={Sekundärquellen}]
%\printbibliography
\end{document}